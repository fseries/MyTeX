\exercice Reprenons la fonction $h$ évoquée dans l'activité
\vspace*{-1em}
\begin{center}
\begin{tikzpicture}[scale=0.5,yscale=3,every node/.style={scale=0.8}]
\tkzInit[xmax=24,xstep=1,ymax=3,ystep=1]
\tkzGrid[sub,
		subxstep=0.25,
		color=gray]
\tkzLabelX
\tkzDrawX[label={\textit{heure le 28/10/2015}},above left=18pt,fill=white]
\tkzLabelY
\tkzDrawY[label={\textit{hauteur d'eau du Gardon d'Al\`es}},below right=8pt]
\draw plot[smooth] file {ales.table.suite};
\tkzAxeX[label=$t$,right=10pt]
\tkzAxeY[label=$h(t)$]
\tkzText(11.8,2.2){$\mathcal{C}_{h}$}
\end{tikzpicture}
\end{center}

\begin{enumerate}
	\item Résoudre graphiquement $h(t) > 2$.
	\item Déterminer graphiquement les antécédents de $2$ par $h$.
	\item Déterminer graphiquement $h(22)$.
	\item Déterminer graphiquement le maximum de la fonction $h$ et la valeur de $t$ pour laquelle on l'obtient.
	\item Déterminer graphiquement les valeurs de~$t$ pour lesquelles la fonction~$h$ est croissante et celles où la fonction~$h$ est décroissante.
	\item Exprimer les réponses de la question précédente sous forme d'intervalles de valeurs de $t$.
	\item Pour chacune des questions précédentes, exprimer en langage courant ce qui est demandé.
	\item Répondre maintenant aux questions de 1. à \addtocounter{enumi}{-2}\theenumi \addtocounter{enumi}{2}.
\end{enumerate}