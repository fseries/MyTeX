\exercice On appelle $g$ la fonction représentée ci-dessous.\\ Les propositions suvantes sont-elles vraies ou fausses ?

\begin{enumerate}
	\item Quel que soit $x$ appartenant à $[0,10]$, le nombre $g(x)$ est positif ou nul.\footnote{\textbf{Remarque} Cette proposition pourrait s'écrire comme ceci en langage mathématique~: $\forall x \in [0,10],\quad g(x) \geqslant 0$}
	\item Quel que soit $x$ appartenant à $[0,10]$, $g(x)$ est plus grand que 1.
	\item $g(3) \geqslant 0,04$
	\item $g(2) \geqslant 0,1$
	\item $g(2) \geqslant g(1)$
\end{enumerate}
\begin{center}
	\begin{tikzpicture}[scale=0.9]%,xscale=1,yscale=10,every node/.style={scale=0.75}]
\tkzSetUpPoint[shape=cross]%,size=20pt,color=teal,fill=teal]
\tkzInit[xmin=0,xmax=10, ymin=0,ymax=0.14,xstep=1,ystep=0.02]
\tkzGrid[sub,subxstep=1,subystep=0.01]
\tkzAxeXY
%\tkzFct[smooth,samples=200,domain = 0:10]{exp(- exp(((0 - \x)/(2))))}
% Densité de proba de la loi de Gumbel avec \beta = 3
\tkzFct[smooth,samples=200,domain = 0:10]{\x*exp(- \x)/3}
\tkzDefPointByFct[draw,ref=A](3)
\tkzLabelPoint[above right](A){$A$}
\tkzCrossPoint{A}
\tkzDefPointByFct[draw,ref=B](10)
\tkzLabelPoint[above right](B){$B$}
\tkzCrossPoint{B}
%\tkzDefPointByFct[draw,ref=C](0.27229)
%\tkzLabelPoint[red,above right](C){$C$}
%\tkzCrossPoint{C}
%\tkzFct[red,samples=2,domain = 0:10]{-0.007142*\x + 0.07142}
%\tkzText(8.5,2.5){$\mathcal{C}_{g}$}
%\tkzText(1.5,2.5){$\mathcal{C}_{f}$}
\end{tikzpicture}
\end{center}
Ensuite,
\begin{enumerate}
	\setcounter{enumi}{5}
	\item Tracez la droite $\mathcal{D}$ passant par les points $A$ et $B$, elle coupe la courbe représentative de $g$ en $C$.
	\item On notera $C\left(c,~g\left(c\right)\right)$, déterminez l'intervalle sur lequel la fonction~$g$ est au dessus de la droite~$\mathcal{D}$.
	\item Déterminez l'intervalle (ou la réunion d'intervalles) sur lequel (laquelle) la fonction~$g$ est au dessus de la droite~$\mathcal{D}$.
	\item Déterminez l'expression de la fonction affine $d$ en fonction de $x$ représentée par la droite~$\mathcal{D}$.
\end{enumerate}