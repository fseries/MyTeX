\exercice~\\
Soit la fonction $f$ définie sur $\mathbb{R}$ par : $f(x) = 4x^2 + 24x + 35$ (\textbf{Expression~1})

On donne d’autres expressions possibles de la fonction $f$~:

\begin{enumerate}[~~~~]
	\vspace*{\stretch{1}}
	
	\item \textbf{Expression~2} $4(x + 3)^2 - 1$
	
	\vspace*{\stretch{1}}
	
	\item \textbf{Expression~3} $(2x + 5)(2x + 7)$
	
	\vspace*{\stretch{1}}
\end{enumerate}

\begin{enumerate}
	\item En développant les expressions 2 et 3, vérifier que toutes deux représentent bien la fonction $f$.
	\item Dans chaque situation proposée ci-dessous, en choisissant l’expression la plus appropriée, répondre à la question posée :
	\begin{enumerate}
		\item Déterminer l’image par $f$ de $-1$.
		\item Déterminer $f(-3)$.
		\item Déterminer les antécédents de $35$.
		\item Déterminer l’image de $\sqrt{2}$.
		\item Déterminer les antécédents de $-1$.
	\end{enumerate}
\end{enumerate}	 

\vspace*{\stretch{1}}

\exerciceprime~\\
Soit la fonction $g$ définie sur $\mathbb{R}$ par : $g(x) = 2x^2 + 7x - 4$ (\textbf{Expression~1})

On donne d’autres expressions possibles de la fonction $g$.
\begin{enumerate}
	\item Parmi les expressions suivantes indiquer en justifiant par un développement celles qui représentent aussi la fonction $g$.

	\vspace*{\stretch{1}}

	\textbf{Expression~2} $(2x - 1)(x + 4)$

	\vspace*{\stretch{1}}

	\textbf{Expression~3} $(2x - 1)(x + 1) + 3(2x - 1)$

	\vspace*{\stretch{1}}

	\textbf{Expression~4} $2(x + 3)^2 - 3x - 22$
	
	\vspace*{\stretch{1}}	
	
	\item Dans chaque situation proposée ci-dessous, en choisissant l’expression la plus appropriée qui est effectivement une expression de $g(x)$, répondre à la question posée :
	\begin{enumerate}
		\item Déterminer l’image par $g$ de $-1$.
		\item Déterminer $g(-4)$.
		\item Déterminer les antécédents de $0$ pour la fonction $g$.
		\item Déterminer l’image de $\sqrt{2}$ par la fonction $g$.
	\end{enumerate}
\end{enumerate}

\vspace*{\stretch{1}}
