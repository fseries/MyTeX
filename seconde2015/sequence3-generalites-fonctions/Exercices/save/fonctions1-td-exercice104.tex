\exercice Abonnement à la piscine

\vspace*{\stretch{1}}

Anna, Tom et John veulent aller à la piscine cette année. La piscine qu'ils ont choisie propose les tarifs suivants :

\vspace*{\stretch{1}}

\begin{enumerate}[~]
	\item \textbf{Tarif 1} L'entrée à \EUR{3,50}.
	\item \textbf{Tarif 2} Un abonnement de \EUR{50} donnant accès à la salle + \EUR{2} par entrée.
	\item \textbf{Tarif 3} Un abonnement de \EUR{180} pour un accès illimité pendant un an.
\end{enumerate}

\vspace*{\stretch{1}}

Ils ont tous les trois une pratique différente et plus ou moins intense de la natation. On se propose de leur indiquer quel est l'abonnement le plus avantageux pour chacun d'entre eux en fonction de leur fréquentation de la piscine. Tom estime qu'il ira environ 2 fois par mois à la piscine. Anna compte se rendre à la piscine une fois par semaine. Pour maintenir son niveau national, John ira au moins deux fois par semaine.

\vspace*{\stretch{1}}

\begin{enumerate}[1)~~~]
\item Quel est le prix pour 20 entrées dans les trois cas ?

\vspace*{\stretch{1}}

\item Exprimer les fonctions $f_1(x)$, $f_2(x)$ et $f_3(x)$ qui représentent respectivement le prix à payer en fonction de \mbox{$x=$ \og nombre de séances\fg} avec les tarifs 1, 2 et 3.

%\fbox{$f_1(x)=$~~~~~~~~~~~~~~~~~~~~~~~~~~~~~~~~~~~~~~}
%
%\fbox{$f_2(x)=$~~~~~~~~~~~~~~~~~~~~~~~~~~~~~~~~~~~~~~}
%
%\fbox{$f_3(x)=$~~~~~~~~~~~~~~~~~~~~~~~~~~~~~~~~~~~~~~}

\vspace*{\stretch{1}}

\item Reproduire et remplir le tableau suivant avec le prix à payer en fonction du nombre d'entrées pour chaque tarif.
	\begin{center}
		\begin{tabular}{|>{\centering}m{100pt}|>{\centering}m{30pt}|>{\centering}m{30pt}|>{\centering}m{30pt}|>{\centering}m{30pt}|m{30pt}|}
		\hline 
	%			&  &   &   &   & \\
		Nombre d'entrées & $10$ & $20$ & $30$ & $40$ & $~~50$ \\
	%			&  &   &   &   & \\
		\hline
	%			&  &   &   &   &	\\
		Prix 1 	&  &   &   &   &	\\ 
	%			&  &   &   &   &	\\
		\hline 
	%			&  &   &   &   &	\\
		Prix 2 	&  &   &   &   &	\\ 
	%			&  &   &   &   &	\\
		\hline 
	%			&  &   &   &   &	\\
		Prix 3 	&  &   &   &   &	\\ 
	%			&  &   &   &   &	\\
		\hline 
		\end{tabular}~\\~\\
	\end{center}
	
\vspace*{\stretch{1}}

\item Tracer la représentation graphique des fonctions $f_1(x)$, $f_2(x)$ et $f_3(x)$. \textbf{Attention aux unités !} On prendra $2~cm=10$ entrées sur l'axe des abscisses et $1~cm=~$\EUR{20} sur l'axe des ordonnées.

\vspace*{\stretch{1}}

\item Lire graphiquement~:
		\begin{enumerate}
			\item l'image de 50 par $f_1~$	% : ~~\fbox{$f_1(50)=$~~~~~~~~~~~~~~~~~~}
			\item l'image de 50 par $f_2~$	% : ~~\fbox{$f_2(50)=$~~~~~~~~~~~~~~~~~~}
			\item l'image de 50 par $f_3~$	% : ~~\fbox{$f_3(50)=$~~~~~~~~~~~~~~~~~~}
			\item $\Longrightarrow$ Quel est le tarif le plus intéressant pour 50~entrées ?	% \fbox{Tarif~~~~~}
		\end{enumerate}

\vspace*{\stretch{1}}

\newpage

\item $f_1(x)=70$
		\begin{enumerate}
			\item Mathématiquement, $x$ est l'\textbf{antécédent} de 70.\\ Résoudre graphiquement $f_1(x)=70$.	% : \fbox{\parbox[c][1em][c]{0.15\textwidth}{{$x=$}}}
			\item Que représente concrètement $x$ dans ce cas ?
		\end{enumerate}
		
%\vspace*{\stretch{1}}

\item Résoudre graphiquement $f_1(x)=f_2(x)$.	% : \fbox{\parbox[c][1em][c]{0.15\textwidth}{{$x=$}}}

%\vspace*{\stretch{1}}

\item Déterminer quels tarifs doivent choisir Anna, Tom et John.	% \fbox{Anna : ~~~}~~\fbox{Tom : ~~~}~~\fbox{John : ~~~}

%\vspace*{\stretch{8}}

%\item Retrouver par le calcul le nombre de séances à partir duquel il est plus avantageux de choisir le tarif~2.
%\item Même question pour le tarif 3.
\end{enumerate}

%\begin{center}
%	\begin{tabular}{|>{\centering}m{100pt}|>{\centering}m{30pt}|>{\centering}m{30pt}|>{\centering}m{30pt}|>{\centering}m{30pt}|m{30pt}|}
%	\hline 
%					&  &   &   &   & \\
%	Nombre d'entrées& $10$ & $20$ & $30$ & $40$ & $~~50$ \\
%					&  &   &   &   & \\
%	\hline
%					&  &   &   &   &	\\
%	Prix 1 			& $35$ & $70$ & $105$ & $140$ & $~175$		\\ 
%	$f_1(x)=3,5x$	&  &   &   &   &	\\
%	\hline 
%					&  &   &   &   &	\\
%	Prix 2 			& $70$ & $90$ & $110$ & $130$ & $~150$		\\ 
%	$f_2(x)=2x+50$	&  &   &   &   &	\\
%	\hline 
%					&  &   &   &   &	\\
%	Prix 3		 	& $180$ & $180$ & $180$ & $180$ & $~180$	\\
%	$f_3(x)=180$	&  &   &   &   &	\\ 
%	\hline 
%	\end{tabular}~\\~\\
%\end{center}

%\begin{center}
%	\begin{tabular}{|>{\centering}m{100pt}|>{\centering}m{30pt}|>{\centering}m{30pt}|>{\centering}m{30pt}|>{\centering}m{30pt}|m{30pt}|}
%	\hline 
%%			&  &   &   &   & \\
%	Nombre d'entrées & $10$ & $20$ & $30$ & $40$ & $~~50$ \\
%%			&  &   &   &   & \\
%	\hline
%%			&  &   &   &   &	\\
%	Prix 1 	&  &   &   &   &	\\ 
%%			&  &   &   &   &	\\
%	\hline 
%%			&  &   &   &   &	\\
%	Prix 2 	&  &   &   &   &	\\ 
%%			&  &   &   &   &	\\
%	\hline 
%%			&  &   &   &   &	\\
%	Prix 3 	&  &   &   &   &	\\ 
%%			&  &   &   &   &	\\
%	\hline 
%	\end{tabular}~\\~\\
%\end{center}

%%
%% Graphique
%%
%%
%%\clearpage
%%\begin{rotate}{-90}
%%\begin{tikzpicture}[
%%						xscale=7,
%%						yscale=1.7,
%%						scale=0.04,
%%						every node/.style={scale=1}
%%					]
%%	\coordinate(O) at (0,0);
%%	\draw (O) node [below left] {$O$};
%%	\draw [->,>=stealth, very thick] (-4,0)--(90,0);
%%	\draw [->,>=stealth, very thick] (0,-30)--(0,220);
%%	\draw [dashed,red] (100/3,0)--(100/3,350/3);
%%	\draw (100/3,0) node [below] {$\dfrac{100}{3}$};
%%	\draw [dashed,red] (65,0)--(65,180);
%%	\draw (0,180) node [left] {$180$};
%%	\foreach \nombre in {10, 20, ..., 70, 65}
%%		\draw (\nombre,-5)
%%			node [below] {\nombre}
%%			node [above] {$|$};
%%%	\draw [dashed,red] (0,350/3)--(100/3,350/3);
%%%	\draw (0,350/3) node [left] {$\dfrac{350}{3} \approx 116,66$};
%%%	\draw [->,>=latex,dashed,red] (1,-1)--(1,25);
%%%	\draw [->,>=latex,dashed,red] (1,-1)--(1,12.5);
%%%	\draw [->,>=latex,dashed,red] (1,25)--(0,25);
%%%	\draw (1,-1) node [below] {$x$};
%%%	\draw (-0.1,25) node [left] {$f(x)$};
%%%	\draw [dashed] (5,-1)--(5,25);
%%%	\draw [->,>=stealth, very thick] (0,0)--(1,0);
%%%	\draw [->,>=stealth, very thick] (0,0)--(0,1);
%%	\draw plot [domain=0:60] (\x,{3.5*(\x)}) node [above] {$f_1(x)=3,5x$};
%%	\draw plot [domain=0:70,near end] (\x,{50+(2*\x)}) node [above] {$f_2(x)=2x+50$};
%%	\draw plot [domain=0:80,near end] (\x,{180}) node [below] {$f_3(x)=180$};
%%\end{tikzpicture}~\\~\\
%%\end{rotate}

%\begin{tikzpicture}[xscale=10,scale=0.1,every node/.style={scale=1}]
%	\coordinate(O) at (0,0);
%	\draw (O) node [below left] {$O$};
%	\draw [->,>=stealth, very thick] (-2,0)--(8,0);
%	\draw [->,>=stealth, very thick] (0,-5)--(0,40);
%	\draw [->,>=latex, dashed,red] (-0.1,25)--(1,25);
%	\draw [->,>=latex, dashed,red] (-0.1,25)--(5,25);
%	\draw [->,>=latex, dashed,red] (1,25)--(1,12.5);
%	\draw [->,>=latex, dashed,red] (5,25)--(5,12.5);
%	\draw [->,>=latex, dashed,red] (1,25)--(1,0);
%	\draw [->,>=latex, dashed,red] (5,25)--(5,0);
%	\draw [dashed,red] (1,-1)--(1,25);
%	\draw [dashed,red] (5,-1)--(5,25);
%	\draw (-0.1,25) node [left] {$y$};
%	\draw (1,-1) node [below] {$x_1$};
%	\draw (5,-1) node [below] {$x_2$};
%%	\draw [->,>=stealth, very thick] (0,0)--(1,0);
%%	\draw [->,>=stealth, very thick] (0,0)--(0,1);
%	\draw plot [domain=-1:7] (\x,{2*(\x)^2 - 12*(\x) + 35});
%\end{tikzpicture}
%\hspace*{\stretch{1}}
