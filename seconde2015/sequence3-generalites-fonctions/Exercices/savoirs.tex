\setcounter{exercice}{0}
\setcounter{probleme}{0}
\exercice % 1 
	\begin{enumerate}
		\item Résoudre graphiquement $h(t) > 2$.
		\item Déterminer graphiquement les antécédents de $2$ par $h$.
		\item Déterminer graphiquement $h(22)$.
		\item Déterminer graphiquement le maximum de la fonction $h$ et la valeur de $t$ pour laquelle on l'obtient.
		\item Déterminer graphiquement les valeurs de~$t$ pour lesquelles la fonction~$h$ est croissante et celles où la fonction~$h$ est décroissante.
		\item Exprimer les réponses de la question précédente sous forme d'intervalles de valeurs de $t$.
		\item Pour chacune des questions précédentes, exprimer en langage courant ce qui est demandé.
		\item Répondre maintenant aux questions de 1. à \addtocounter{enumi}{-2}\theenumi \addtocounter{enumi}{2}.
	\end{enumerate}
\exercice % 2
	\begin{enumerate}
		\item Résoudre $x - 2 = -2x + 1$.
		\item Tracer $f$ et $g$ dans un repère d'unité 1 cm. $x \in \left[-1~;~2\right], y \in \left[-3~;~3\right]$
		\item Que signifie $f(x) = g(x)$ ?
		\item Que signifie $f(x) \geqslant g(x)$ ?
	\end{enumerate}
\exercice % 3
\exercice % 4
\exercice % 5
\exerciceprime 
\exercice % 6
\exercice % 7
\exerciceprime

% Fin page 1

\exercice % 8
\exercice % 9
\exercice % 10

% Fin page 2

\exercice % 11
\exercice % 12
\exerciceprime

% Fin page 3

\exercice % 13
\exerciceprime
\exercice % 14

% Fin page 4

\exercice % 15

\probleme % 1
\probleme % 2


% Fin page 5