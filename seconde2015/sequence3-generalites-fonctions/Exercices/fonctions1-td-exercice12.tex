\exercice On appelle $h$ la fonction représentée graphiquement ci-dessous. Résolvez graphiquement~:\\

\begin{minipage}{0.35\linewidth}
	\begin{enumerate}
		\item $h(x) \leqslant 0$ %\\ \rouge{$\mathbf{x \in \left[ - ~;~ \right]}$}\\
		\item $h(x) \leqslant -5$ %\\ \rouge{$\mathbf{x \in \left[ - ~;~ \right]}$}\\
		\item $-5 \leqslant h(x) \leqslant 0$ %\\ \rouge{$\mathbf{x \in \left[ - ~;~ \right]}$}\\
	\end{enumerate}
\end{minipage}
\begin{minipage}{0.5\linewidth}
	\begin{enumerate}
		\setcounter{enumi}{3}
		\item Tracez la droite $d(x)=-x-6$
		\item Résolvez $h(x) \leqslant d(x)$
		\item[~] ~
	\end{enumerate}
\end{minipage}
\begin{center}
	\begin{tikzpicture}[scale=0.75,xscale=1,every node/.style={scale=0.75}]
%\tkzSetUpPoint[shape=cross,size=20pt,color=teal,fill=teal]
\tkzInit[xmin=-7.5,xmax=8.75,xstep=1, ymin=-6,ymax=2]
\tkzGrid[sub,subxstep=1,subystep=1]
\tkzAxeXY
\tkzFct[smooth,samples=100,domain = -7.5:8.75]{((\x)+6)*((\x)-6)/((\x)+8)}
%\tkzFct[samples=2,domain = -7.5:8.75]{-\x-6}
%\tkzDefPointByFct[draw,ref=A](-6)
%\tkzLabelPoint[above right](A){$A$}
%\tkzCrossPoint{A}
%\tkzDefPointByFct[draw,ref=B](-1)
%\tkzLabelPoint[above left](B){$B$}
%\tkzCrossPoint{B}
%\tkzText(8.5,2.5){$\mathcal{C}_{g}$}
\tkzText(1.7,-2.7){$\mathcal{C}_{f}$}
\end{tikzpicture}
\end{center}

\exerciceprime On s'intéresse à la fonction $h$ représentée ci-dessous

\begin{enumerate}
	\item Peut-on dresser le tableau de signes de cette fonction sur $\mathbb{R}$ à l'aide de la courbe représentative dont on dispose à l'exercice précédent~?
	\item Cette fonction a pour expression $h : x \mapsto \dfrac{(x+6)(x-6)}{x+8}$\\
			Dressez les tableaux de signes de la fonction $h$ pour $x \in [-7,20]$, puis~pour~$x \in \mathbb{R}$.
\end{enumerate}