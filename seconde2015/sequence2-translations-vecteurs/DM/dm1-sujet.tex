\exercice Placez le point $B$, image du point $A$ par la translation de vecteur~\vectaffiche{u}~+~\vectaffiche{v}.

\begin{tikzpicture}[scale=0.7]
	\placerpoint{A}{0}{0}{below left};
%	Correction
%	\vectAu{u}{0}{0}{5}{3}{1pt};
%	\vectAu{v}{5}{3}{-1}{-3}{1pt};
	\vectAu{u}{12}{0}{5}{3}{1pt};
	\vectAu{v}{9}{2}{-1}{-3}{1pt};
\end{tikzpicture}

\exercice	\begin{enumerate}
				\item Construisez le point $B$ tel que \vectaffiche{AB} = \vectaffiche{u} + \vectaffiche{v}.
				\item Construisez le point $C$ tel que \vectaffiche{AC} = \vectaffiche{u} + \vectaffiche{v} + \vectaffiche{w}.
				\item Construisez le point $D$ tel que \vectaffiche{AD} = \vectaffiche{u} + \vectaffiche{v} $-$ \vectaffiche{w}.\\
			\end{enumerate}

\begin{tikzpicture}[scale=0.7]
	\placerpoint{A}{0}{0}{below left};
%	Correction
%	\vectAu{u}{0}{0}{10}{3}{1pt};
%	\vectAu{v}{10}{3}{-5}{1}{1pt};
%	\vectAu{w}{5}{4}{-1}{-2}{1pt};

	\vectAu{u}{6}{-2}{10}{3}{1pt};
	\vectAu{v}{4}{5}{-5}{1}{1pt};
	\vectAu{w}{12}{4}{-1}{-2}{1pt};
\end{tikzpicture}


\exercice Relation de Chasles

\begin{enumerate}
	\item \vectaffiche{BC} $+$ \vectaffiche{AB} $=$ 
	\item \vectaffiche{PC} $+$ \vectaffiche{AP} $+$ \vectaffiche{CD} $=$ 
	\item \vectaffiche{AB} $+$ \vectaffiche{BC} $-$ \vectaffiche{AC} $=$ 
	\item \vectaffiche{AC} $-$ \vectaffiche{BC} $+$ \vectaffiche{DA} $=$ 
\end{enumerate}

\exercice Vrai ou faux ?
	\begin{enumerate}
%		\item Dire que $PHYS$ est un parallélogramme équivaut à dire que \vectaffiche{PY} $=$ \vectaffiche{PH} $+$ \vectaffiche{PS}
		\item Dire que $VECT$ est un parallélogramme équivaut à dire que \vectaffiche{VC} $=$ \vectaffiche{VE} $+$ \vectaffiche{CT}
		\item Dire que $VECT$ est un parallélogramme équivaut à dire que \vectaffiche{ET} $=$ \vectaffiche{EC} $+$ \vectaffiche{EV}
		\item Dire que $ROND$ est un parallélogramme équivaut à dire que \vectaffiche{RO} $=$ \vectaffiche{ON} $+$ \vectaffiche{ND}
	\end{enumerate}

\exercice Soit un segment de longueur $l$, on réalise les transformations suivantes~:
\begin{enumerate}
	\item[$*$] On sépare le segment en 3 segments de longueurs égales,
	\item[$*$] On remplace le segment central de longueur $\dfrac{l}{3}$ par un triangle équilatéral de côté $\dfrac{l}{3}$,
	\item[$*$] On supprime la base du triangle ainsi construit (voir ci-dessous).
\end{enumerate}
			
\vspace*{2em}


\begin{center}
\begin{tikzpicture}
	\draw (0,0) -- (9,0)
		node[pos=0.333333]{$|$}
		node[pos=0.666666]{$|$}
		node[pos=0.166666,below]{$\dfrac{l}{3}$}
		node[pos=0.5,below]{$\dfrac{l}{3}$}
		node[pos=0.833333,below]{$\dfrac{l}{3}$};
	\draw (0,-5)
		-- ++ (0:3) %node[midway,above]{$\dfrac{l}{3}$}
		-- ++ (60:3) %node[midway,above left]{$\dfrac{l}{3}$}
		-- ++ (-60:3) %node[midway,above right]{$\dfrac{l}{3}$}
		-- ++ (0:3) %node[midway,above]{$\dfrac{l}{3}$}
		;
	\draw[dotted] (3,1) -- (3,-6);
	\draw[dotted] (6,1) -- (6,-6);
\end{tikzpicture}
\end{center}

\begin{enumerate}
	\item Quelle est la longueur de la ligne ainsi obtenue~?
	\item On recommence une fois le processus sur chaque segment de cette ligne.\\ Quelle est la longueur de la ligne obtenue~?
	\item Peut-on prévoir l'évolution de la longueur si on continue le processus une troisième~fois~? une quatrième fois~? \ldots une centième fois ?
\end{enumerate}

\vspace*{2em}

\hfill \BONNESVACANCES