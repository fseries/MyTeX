\exercice Placez le point $B$, image du point $A$ par la translation de vecteur~\vectaffiche{u}~+~\vectaffiche{v}.\\

\begin{tikzpicture}[scale=0.7]
	\placerpoint{A}{0}{0}{below left};
	\grille{-4}{-2}{15}{7}{help lines}
	\vectAu{u}{9}{2}{5}{4}{1pt};
	\vectAu{v}{0}{2}{-3}{2}{1pt};
%	Correction
%	\vectAuColor{u}{0}{0}{5}{4}{1pt}{0.5}{below right}{red};
%	\vectAuColor{v}{5}{4}{-3}{2}{1pt}{0.5}{above right}{red};
%	\placerpointcolor{B}{2}{6}{above left}{red};
%	\draw (0,0) -- (2,6) node [fill=white,midway,above,sloped,pos=0.6] {\textcolor{red}{$\overrightarrow{u}+\overrightarrow{v}$}};
%	\draw [->,>=latex,line width=1pt,color=red] (0,0) -- (2,6);

\end{tikzpicture}


\exercice
\begin{enumerate}
	\item Nommer 3 vecteurs égaux au vecteur \vectaffiche{AB}.
	%\reponseEX{Réponse~: \vectaffiche{BG}, \vectaffiche{DC}, \vectaffiche{CZ}}
	\item Nommer 3 vecteurs opposés à \vectaffiche{XZ}.
	%\reponseEX{Réponse~: \vectaffiche{ZX}, \vectaffiche{GY}, \vectaffiche{FC}}
	\item Nommer le point image de $E$ par la translation de vecteur \vectaffiche{AB} $+$ \vectaffiche{FY}.
	%\reponseEX{Réponse~: C'est le point $X$.}
	\item Nommer le point image de $C$ par la translation de vecteur \vectaffiche{CB} $+$ \vectaffiche{DE}.
	%\reponseEX{Réponse~: C'est le point $F$.}

\end{enumerate}
\begin{center}
\begin{tikzpicture}[scale=0.8,every node/.style={scale=1}]
	\tikzstyle{vect}=[->,>=stealth,very thick]
	\tikzstyle{vectred}=[vect,red]
	\tikzstyle{etiquette}=[fill=white,
							midway,
							sloped,
							above]
								
\grille{0}{0}{11}{9}{help lines};

\coordinate (A) at (2,5);
\coordinate (B) at (6,6);
\coordinate (C) at (5,3);
\coordinate (D) at (1,2);
\coordinate (E) at (2,1);
\coordinate (F) at (7,5);
\coordinate (G) at (10,7);
\coordinate (H) at (2,7);
\coordinate (X) at (7,2);
\coordinate (Y) at (8,5);
\coordinate (Z) at (9,4);
\foreach \point in {A, ..., H, X, Y, Z}
	\draw (\point) node{$\times$} node[below right]{$\point$};


%\draw[vectred] (A) -- (B);

\end{tikzpicture}
\end{center}

\newpage

\exercice Simplifiez l'écriture des vecteurs suivants en précisant la ou les propriétés utilisées.

\begin{enumerate}
	\item \vectaffiche{AB} $+$ \vectaffiche{BC} $=$ \\
	\item \vectaffiche{AB} $+$ \vectaffiche{CA} $+$ \vectaffiche{BD} $=$ \\
	\item \vectaffiche{AD} $-$ \vectaffiche{AC} $+$ \vectaffiche{DC} $=$ \\
	\item \vectaffiche{AE} $-$ \vectaffiche{BE} $+$ \vectaffiche{FA} $=$ \\
\end{enumerate}

\exercice $PHYS$ est un parallélogramme. Simplifiez l'écriture des vecteurs suivants en précisant la ou les propriétés utilisées.
\begin{enumerate}
	\item \vectaffiche{PH} $+$ \vectaffiche{PS} $=$ \\
	\item \vectaffiche{HY} $+$ \vectaffiche{HP} $=$ \\
	\item \vectaffiche{YS} $+$ \vectaffiche{YH} $+$ \vectaffiche{PY} $=$ \\
	\item \vectaffiche{PY} $-$ \vectaffiche{PS} $-$ \vectaffiche{SH} $=$ \\
\end{enumerate}

%\begin{tikzpicture}[scale=1,every node/.style={scale=1}]
%	\tikzstyle{vect}=[->,>=stealth,line width=3pt]
%	\tikzstyle{vectred}=[vect,red]
%	\tikzstyle{vectgray}=[vect,gray]
%	\tikzstyle{etiquette}=[fill=white,
%				midway,
%				sloped,
%				above]
%%				\coordinate (V) at (0,0);
%%				\coordinate (E) at (4,0);
%%				\coordinate (C) at (5,2);
%%				\coordinate (T) at (1,2);
%	\placerpoint{P}{0}{0}{below left};
%	\placerpoint{H}{4}{0}{below right};
%	\placerpoint{Y}{5}{2}{above right};
%	\placerpoint{S}{1}{2}{above left};
%	\draw (P) -- (H) -- (Y) -- (S)-- cycle;
%%	\draw[vectgray] (V) -- (E);
%%	\draw[vectgray] (C) -- (T);
%%	\draw[vect] (V) -- (C);
%%	\draw (9,1) node {\textcolor{red}{\vectaffiche{VE}} $+$ \vectaffiche{CT} $=$ \textcolor{red}{\vectaffiche{TC}} $+$ \vectaffiche{CT} $=$ \vectaffiche{TT} $=$ \vectaffiche{0} $\ne$ \vectaffiche{VC}};
%%	\draw (9,0) node {L'affirmation est fausse !};
%		\end{tikzpicture}





%\exercice
%\begin{enumerate}
%	\item Construisez le point $B$ tel que \vectaffiche{AB} = \vectaffiche{u} $+$ \vectaffiche{v}.
%	\item Construisez le point $C$ tel que \vectaffiche{AC} = \vectaffiche{u} $-$ \vectaffiche{v} $+$ \vectaffiche{w}.
%	\item Construisez le point $D$ tel que \vectaffiche{AD} = \vectaffiche{u} $+$ \vectaffiche{v} $-$ \vectaffiche{w}.\\
%\end{enumerate}
%
%\begin{tikzpicture}[scale=0.7]
%	\placerpoint{A}{0}{0}{below left};
%%	Correction
%%	\vectAu{u}{0}{0}{10}{3}{1pt};
%%	\vectAu{v}{10}{3}{-5}{1}{1pt};
%%	\vectAu{w}{5}{4}{-1}{-2}{1pt};
%
%	\vectAu{u}{6}{-2}{10}{3}{1pt};
%	\vectAu{v}{4}{5}{-5}{1}{1pt};
%	\vectAu{w}{12}{4}{-1}{-2}{1pt};
%\end{tikzpicture}
