\begin{minipage}{0.45\textwidth}
\thispagestyle{sujet2}

\vspace*{1em}

\exercice\hfill \emph{4 points}

Un élève a voulu tracer le parallélogramme $DCBA$ mais il ne sait pas où placer le sommet $A$.\\ Comment pouvez-vous l'aider~?\\ Vous détaillerez votre démarche sur votre copie avant de réaliser des calculs.

\begin{center}

\begin{tikzpicture}[scale=1,every node/.style={scale=1}]

%\coordinate (X) at (-0.5,-0.5);
%\coordinate (Y) at (13,13);
%\clip (X) rectangle (Y);
%\placerpoint{A}{-5}{-1}{below left};
\placerpoint{B}{-1}{2}{above left};
\placerpoint{C}{8}{4}{above left};
\placerpoint{D}{4}{1}{above right};
%\placerpoint{M}{6}{4}{above left};
%\draw[very thick,red] (A) -- (B) -- (C) -- (D) -- cycle;
\repereOIJ{-2}{9}{-3}{5};
\foreach \r in {5}
    \draw[thick, below right] (\r,0) node{\r};
%\foreach \r in {5}
%    \draw[thick, above left] (0,\r) node{\r};

\end{tikzpicture}

\end{center}


\vspace*{2em}

\exercice\hfill \emph{6 points}

On se place dans le repère~\emph{orthonormé}\rep{O}{I}{J}. Dans ce repère, on considère les points $\pointcoord{R}{-1}{2}$, $\pointcoord{S}{3}{-2}$, $\pointcoord{T}{6}{5}$ et $\pointcoord{U}{7}{7}$.\\Vos réponses seront argumentées.

	\begin{enumerate}
		\item Le point $R$ appartient-il au cercle $\mathscr{C}$ de centre $T$ passant par~$S$~?
		\item Le point $U$ appartient-il à la médiatrice de $\left[RS\right]$~?
%		\item Déterminer les coordonnées de $Z$ pour que $AZBC$ soit un parallélogramme.
	\end{enumerate}



\vspace{-2em}


\end{minipage}
\newpage

\vspace*{1em}

\begin{minipage}{0.45\textwidth}
\thispagestyle{sujet2}

\vspace*{1em}

%\paragraph{Exercice~4} 
%
%On se place dans le repère \emph{orthonormé}\rep{O}{I}{J}. $ABCD$ est un rectangle de centre $O$. $A$ est tel que son abscisse est négative et son ordonnée positive. Le côté $AB$ est parallèle à l'axe des ordonnées. Pour chaque question, faire une figure en prenant $AB = 4 cm$ et $AD = 6 cm$. Donner les coordonnées des points $A$, $B$, $C$, $D$, $O$, $I$ et $J$~:
%	\begin{enumerate}
%		\item Dans le repère\rep{O}{I}{J}
%		\item Dans le repère\rep{B}{C}{A}
%		\item Dans le repère\rep{A}{B}{D}
%	\end{enumerate}

%\vspace{-2em}

\exercice\hfill \emph{10 points}

Le plan $\left(\mathscr{P}\right)$ est muni d'un repère~\emph{orthonormé}\rep{O}{I}{J}.\\ Dans ce repère, on considère les points suivants $\pointcoord{A}{-9}{-2}$ ; $\pointcoord{B}{-1}{4}$, $\pointcoord{C}{5}{12}$ et $\pointcoord{D}{-3}{6}$.
On désigne par $E$, $F$ et $G$ les milieux respectifs des segments $\left[AB\right]$,  $\left[BC\right]$ et $\left[AC\right]$.

\begin{enumerate}
	\item Démontrez que le triangle $ABC$ est isocèle en $A$.
	\item Démontrez que le quadrilatère $ABCD$ est un losange.
	\item Calculez les coordonnées des points $E$, $F$ et $G$ dans le repère\rep{O}{I}{J}.
	\item Quelle est la nature du triangle $EFG$~? Justifiez.
	\item Déterminez les coordonnées des points $A$, $B$, $C$, $D$, $E$, $F$ et $G$ dans le~repère\rep{A}{B}{C}.
	\item Pourquoi ne peut-on pas calculer des distances dans le~repère\rep{A}{B}{C}~?
\end{enumerate}

\vspace*{1em}

\vspace*{\stretch{1}}
\begin{center}
\begin{tikzpicture}[scale=2,transform shape]
\draw (0,0) \Fin;
\draw (-1em,-1ex) -- (1em,-1ex);
\path[scope fading=south] (-1em,-0.25em) rectangle (1em,-3.75ex);
\draw[yscale=-1] (0,2ex) \Fin;
\end{tikzpicture}
\end{center}
\vspace*{\stretch{1}}

\vspace*{2em}

\hrulefill
\exercice\hfill \emph{Bonus}

Les diagonales d’un quadrilatère $ABCD$ se coupent en $E$.

$I$, $J$, $K$, $L$ sont les milieux respectifs de $[AB]$, $[BC]$, $[CD]$, $[DA]$.

\begin{enumerate}
	\item Faire une figure en y reportant toutes les informations de l’énoncé.
	\item Démontrer que $IJKL$ est un parallélogramme.
\end{enumerate}


\end{minipage}