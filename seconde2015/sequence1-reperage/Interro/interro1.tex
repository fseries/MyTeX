\documentclass[a4paper,10pt,twocolumn,landscape]{article}

\usepackage{superpack2015}

\usepackage[heightrounded]{geometry}	% heightrounded permet d'afficher les footers correctement
\geometry{hmargin=0.5cm,vmargin=1.5cm}

\setlength{\columnseprule}{0.1pt}		% Ligne séparatrice milieu document
\setlength{\columnsep}{20pt}			% Espace de chaque côté de la ligne
\setlength{\headsep}{10pt}
\addtolength{\textheight}{15pt}
%\setlength{\textwidth}{770pt}
%\setlength{\hoffset}{20pt}
\setlength{\voffset}{0pt}

\classichf
	% Nom du style
	{premierepage}
	% Hauteur sous header
	% 14.5pt si une ligne (1 \baselineskip)
	% 29.0pt si deux lignes (2 \baselineskip)
	{29pt}
	% Head
	{\begin{minipage}{0.5\textwidth}\begin{center}\textbf{Mathématiques\\Questions de cours}\end{center}\end{minipage}}
	{}
	{\begin{minipage}{0.5\textwidth}\begin{center}\textbf{Mathématiques\\Questions de cours}\end{center}\end{minipage}}
	% Foot
	{}
	{}
	{}

%\usepackage{showframe}
%\usepackage{layout}

%\usepackage{siunitx}

\begin{document}
\pagestyle{premierepage}	%\thispagestyle{premierepage} pour isoler des styles de pages

Soient $\pointcoord{A}{5}{3}$ et $\pointcoord{B}{4}{6}$ deux points du plan placés dans un repère\rep{O}{I}{J}.
\question Déterminer les coordonnées de $M$ milieu de $[AB]$.\\[1em]
\begin{tikzpicture}
	\path (0,0) -- (10,0)
		node[pos=0]{$x_M = $}
		node[pos=0.6]{$y_M = $}
		node[yshift=-1cm,pos=1.2]{$\pointcoord{M}{~~~~~}{~~~~~}$};
	\draw[dotted] (5,0.3) -- (5,-3);
	\draw[dotted] (10,0.3) -- (10,-3);
	\draw[dotted] (0,-3) -- (10,-3);
\end{tikzpicture}
\question Déterminer la longueur $AB$.\\[1em]
$AB = \sqrt{\parbox[c][2em][c]{0.2\textwidth}~}$\\[1em]
$AB = $\\[1em]
$AB = $\\[1em]
$AB = $\\[1em]																\vspace*{\stretch{1}}\\
.\dotfill
\question Par quoi le vecteur \vectaffiche{AB} est-il défini~?  (3~informations)\\
\begin{enumerate}
	\item ~\\
	\item ~\\
	\item ~\\	
\end{enumerate}
\question Entourer le ou les vecteur(s) opposé(s) à \vectaffiche{AB}.\\[1em]
\begin{inparaenum}[~]
	\item \vectaffiche{AB}\hspace*{\stretch{1}}
	\item $-$\vectaffiche{AB}\hspace*{\stretch{1}}
	\item $-$\vectaffiche{BA}\hspace*{\stretch{1}}
	\item \vectaffiche{BA}\hspace*{\stretch{1}}~
\end{inparaenum}%																\vspace*{\stretch{1}}
\question Entourer le ou les vecteur(s) égal (égaux) à \vectaffiche{AB}.\\[1em]
\begin{inparaenum}[~]
	\item \vectaffiche{AB}\hspace*{\stretch{1}}
	\item $-$\vectaffiche{AB}\hspace*{\stretch{1}}
	\item $-$\vectaffiche{BA}\hspace*{\stretch{1}}
	\item \vectaffiche{BA}\hspace*{\stretch{1}}~
\end{inparaenum}%																\vspace*{\stretch{1}}

\newpage

\setcounter{question}{0}
Soient $\pointcoord{A}{7}{8}$ et $\pointcoord{B}{3}{6}$ deux points du plan placés dans un repère\rep{O}{I}{J}.
\question Déterminer les coordonnées de $M$ milieu de $[AB]$.\\[1em]
\begin{tikzpicture}
	\path (0,0) -- (10,0)
		node[pos=0]{$x_M = $}
		node[pos=0.6]{$y_M = $}
		node[yshift=-1cm,pos=1.2]{$\pointcoord{M}{~~~~~}{~~~~~}$};
	\draw[dotted] (5,0.3) -- (5,-3);
	\draw[dotted] (10,0.3) -- (10,-3);
	\draw[dotted] (0,-3) -- (10,-3);
\end{tikzpicture}
\question Déterminer la longueur $AB$.\\[1em]
$AB = \sqrt{\parbox[c][2em][c]{0.2\textwidth}~}$\\[1em]
$AB = $\\[1em]
$AB = $\\[1em]
$AB = $\\[1em]																\vspace*{\stretch{1}}\\
.\dotfill
\question Par quoi le vecteur \vectaffiche{AB} est-il défini~?  (3~informations)\\
\begin{enumerate}
	\item ~\\
	\item ~\\
	\item ~\\	
\end{enumerate}
\question Entourer le ou les vecteur(s) opposé(s) à \vectaffiche{AB}.\\[1em]
\begin{inparaenum}[~]
	\item \vectaffiche{BA}\hspace*{\stretch{1}}
	\item $-$\vectaffiche{BA}\hspace*{\stretch{1}}
	\item $-$\vectaffiche{AB}\hspace*{\stretch{1}}
	\item \vectaffiche{AB}\hspace*{\stretch{1}}~
\end{inparaenum}%																\vspace*{\stretch{1}}
\question Entourer le ou les vecteur(s) égal (égaux) à \vectaffiche{AB}.\\[1em]
\begin{inparaenum}[~]
	\item \vectaffiche{BA}\hspace*{\stretch{1}}
	\item $-$\vectaffiche{BA}\hspace*{\stretch{1}}
	\item $-$\vectaffiche{AB}\hspace*{\stretch{1}}
	\item \vectaffiche{AB}\hspace*{\stretch{1}}~
\end{inparaenum}%																\vspace*{\stretch{1}}


\end{document}


%
%%\exercicebareme{4}
%\exercicebareme{6}
%\exercicebareme{10}
%\exerciceunpoint
%\exercicebonus
%\FIN
%\hrulefill
