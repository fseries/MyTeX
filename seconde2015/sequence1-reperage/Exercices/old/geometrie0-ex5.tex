\documentclass[a4paper,12pt,twocolumn,landscape]{article}

\usepackage{FabZ}
\usepackage{vecteurs}
\usepackage{repere}

\usepackage{geometry}
\geometry{hmargin=0.5cm,vmargin=1.5cm}

\newcommand{\milieu}[6]{$\pointcoord{A_{\theenumi}}{#1}{#2}$ et $\pointcoord{B_{\theenumi}}{#3}{#4}$ \hfill \reponseEX{$\pointcoord{M_{\theenumi}}{#5}{#6}$}}

\newcommand{\extremite}[6]{$\pointcoord{A_{\theenumi}}{#1}{#2}$ et $\pointcoord{M_{\theenumi}}{#3}{#4}$ \hfill \reponseEX{$\pointcoord{B_{\theenumi}}{#5}{#6}$}}

\newcommand{\longueur}[5]{$\pointcoord{A_{\theenumi}}{#1}{#2}$ et $\pointcoord{B_{\theenumi}}{#3}{#4}$ \hfill \reponseEX{Longueur $A_{\theenumi}B_{\theenumi}$ = $#5$}}

\newcommand{\quadrilatere}[9]{$\pointcoord{A_{\theenumi}}{#1}{#2}$, $\pointcoord{B_\theenumi}{#3}{#4}$ $\pointcoord{C_{\theenumi}}{#5}{#6}$, $\pointcoord{D_\theenumi}{#7}{#8}$ \\ \textcolor{red}{$A_{\theenumi}B_{\theenumi}C_{\theenumi}D_{\theenumi}$ est un #9.}}
%
%\setlength{\headheight}{0pt}
%\pagestyle{fancyplain}
%\fancyhf{}
%\lhead[]{\textbf{TD Vecteurs}}
%\chead[]{}
%\rhead[]{}
%
%\lfoot[]{}
%\cfoot[]{}
%\rfoot[]{}
%%\rfoot[]{Page \thepage~ sur \pageref{LastPage}}

\fancypagestyle{firststyle}
{
	\setlength{\headheight}{0em}
	\fancyhf{}
	\lhead[]{}
	\chead[]{\textbf{Repérage dans le plan}}
	\rhead[]{}

	\lfoot[]{}
	\cfoot[]{}
	\rfoot[]{}
%	\rfoot[]{Page \thepage~ sur \pageref{LastPage}}
}

\fancypagestyle{vide}
{
	\setlength{\headheight}{0em}
	\pagestyle{fancyplain}
	\def\headrulewidth{0em}
	\fancyhf{}
	\lhead[]{}
	\chead[]{}
	\rhead[]{}
	
	\lfoot[]{}
	\cfoot[]{}
	\rfoot[]{}
%	\rfoot[]{Page \thepage~ sur \pageref{LastPage}}
}

\usetikzlibrary{fadings}
\newcommand{\Fin}{node[xshift=-1.5ex,rotate=10]{F}
node[rotate=170]{i}
node[xshift=1.5ex,rotate=45]{n}}

\newcommand{\bonnesvacances}{node[xshift=0ex,rotate=10]{B}
node[xshift=1*1.5ex,rotate=170]{o}
node[xshift=2*1.5ex,rotate=45]{n}
node[xshift=3*1.5ex,rotate=128]{n}
node[xshift=4*1.5ex,rotate=75]{e}
node[xshift=5*1.5ex,rotate=130]{s}
node[xshift=6*1.5ex,rotate=85]{~}
node[xshift=7*1.5ex,rotate=128]{v}
node[xshift=8*1.5ex,rotate=43]{a}
node[xshift=9*1.5ex,rotate=4]{c}
node[xshift=10*1.5ex,rotate=145]{a}
node[xshift=11*1.5ex,rotate=5]{n}
node[xshift=12*1.5ex,rotate=25]{c}
node[xshift=13*1.5ex,rotate=105]{e}
node[xshift=14*1.5ex,rotate=45]{s}
}


\begin{document}
\begin{minipage}{0.45\textwidth}
\thispagestyle{firststyle}

\vspace*{1em}

\paragraph{Exercice~5} Un élève a voulu tracer un parallélogramme $ABCD$ mais il ne sait pas où placer le sommet $C$. Comment peux-tu l'aider~?

\begin{center}

\begin{tikzpicture}[scale=0.5,every node/.style={scale=0.5}]

%\coordinate (X) at (-0.5,-0.5);
%\coordinate (Y) at (13,13);
%\clip (X) rectangle (Y);
\placerpoint{A}{1}{1}{below left};
\placerpoint{B}{10}{3}{below right};
\placerpoint{D}{2}{5}{above left};
\placerpoint{M}{6}{4}{above left};
%\placerpoint{C}{11}{7}{below right};
%\draw[very thick,red] (A) -- (B) -- (C) -- (D) -- cycle;
\repereOIJ{-1}{16}{-1}{11};
\foreach \r in {5,10,15}
    \draw[thick, below right] (\r,0) node{\r};
\foreach \r in {5,10}
    \draw[thick, above left] (0,\r) node{\r};

\end{tikzpicture}

\end{center}

\paragraph{Exercice~6} Pour chaque couple de points $\left(A_i, B_i\right)$ déterminer la longueur $A_iB_i$ du segment $[A_iB_i]$ (arrondir au centième par défaut si nécessaire)~:

\begin{enumerate}
	\item \longueur{2}{1}{0}{8}{7,28}
	\item \longueur{-50}{20}{10}{0}{63,24}
	\item \longueur{2}{1}{5}{5}{5}
	\item \longueur{-20}{30}{50}{10}{72,80}
	\item \longueur{6}{2}{-2}{6}{8,94}
	\item \longueur{2}{-2}{-6}{4}{10}
	\item \longueur{-2}{-4}{4}{-2}{6,32}
	\item \longueur{-20}{0}{20}{30}{50}
	\item \longueur{-60}{-20}{-20}{-60}{56,56}
	
	\item Calculer la longueur $A_4B_2$
\end{enumerate}
%
%\begin{center}
%
%\begin{tikzpicture}[scale=0.5,every node/.style={scale=0.5}]
%
%%\coordinate (X) at (-0.5,-0.5);
%%\coordinate (Y) at (13,13);
%%\clip (X) rectangle (Y);
%\placerpoint{A}{1}{1}{below left};
%\placerpoint{B}{10}{3}{below right};
%\placerpoint{D}{2}{5}{above left};
%\placerpoint{M}{6}{4}{above left};
%%\placerpoint{C}{11}{7}{below right};
%%\draw[very thick,red] (A) -- (B) -- (C) -- (D) -- cycle;
%\repere{-1}{16}{-1}{11};
%\foreach \r in {5,10,15}
%    \draw[thick, below right] (\r,0) node{\r};
%\foreach \r in {5,10}
%    \draw[thick, above left] (0,\r) node{\r};
%
%\end{tikzpicture}
%
%\end{center}

\vspace{-2em}

\end{minipage}
\newpage
\begin{minipage}{0.45\textwidth}
\thispagestyle{firststyle}

\vspace*{1em}

\paragraph{Exercice~7} Les points $A_i$, $B_i$, $C_i$ et $D_i$ suivants, constituent les quadrilatères $A_iB_iC_iD_i$. On cherche à déterminer la nature de chacun d'eux.

\begin{enumerate}
	\item \quadrilatere{2}{2}{8}{4}{12}{8}{6}{6}{parallélogramme}
	\item \quadrilatere{1}{-1}{4}{5}{6}{2}{9}{1}{quadrilatère quelconque}
	\item \quadrilatere{-2}{-2}{6}{0}{14}{8}{6}{6}{parallélogramme}
	\item \quadrilatere{-4}{4}{2}{-2}{6}{2}{0}{8}{rectangle}
	\item \quadrilatere{0}{2}{2}{-6}{10}{-4}{8}{4}{carré}
	\item \quadrilatere{-3}{-4}{4}{-3}{5}{4}{-2}{3}{losange}
\end{enumerate}


\paragraph{Exercice~8} 

\begin{enumerate}
	\item 
\end{enumerate}

\vspace{-2em}

\end{minipage}
\end{document}
%
%
%\paragraph{Exercice~2} Propriétés des quadrilatères
%
%\begin{tikzpicture}[scale=1,every node/.style={scale=0.7}]
%\tikzstyle{debutfin}=[ellipse,draw,text=red]
%\tikzstyle{instruct}=[rectangle,draw,fill=yellow!50]
%\tikzstyle{test}=[diamond, aspect=6,thick,
%draw=blue,fill=yellow!50,text=blue]
%\tikzstyle{es}=[rectangle,draw,rounded corners=4pt,fill=blue!25]
%
%\node[debutfin] (debut) at (0,3) {Début};
%\node[es] (lire) at (0,2) {Prendre un quadrilatère $Q$};
%\node[test] (test) at (0,0) {Les diagonales ont-elles un milieu commun \ ?};
%%\node[instruct] (init) at (-2,2.5) {$S\leftarrow 0$};
%\node[instruct] (plus) at (0,-2.5) {$S\leftarrow S+N$};
%\node[instruct] (moins) at (0,-3.5) {$N\leftarrow N-1$};
%\node[es] (afficher) at (-4,-2) {Afficher la somme $S$};
%\node[debutfin] (fin) at (-4,-3) {Fin};
%
%\tikzstyle{suite}=[->,>=stealth,thick,rounded corners=4pt]
%\draw[suite](debut) -- (lire);
%\draw[suite](lire) -- (test.north);
%%\draw[suite](init) -- (test.north);
%\draw[suite](test.south) -- (plus);
%\draw[suite](plus) -- (moins);
%\draw[suite](test) -| (afficher);
%\draw[suite](afficher) -- (fin);
%\end{tikzpicture}
%
%
%%%%%\begin{tikzpicture}
%%%%%% définition des styles
%%%%%\tikzstyle{quadri}=[rectangle,draw,fill=yellow!50,text=blue]
%%%%%\tikzstyle{estun}=[->,>=latex,very thick,dotted]
%%%%%% les nœuds
%%%%%\node[quadri] (Q) at (0,3) {Quadrilatère};
%%%%%\node[quadri] (P) at (0,1.5) {Parallélogramme};
%%%%%\node[quadri] (R) at (-3,0) {Rectangle};
%%%%%\node[quadri] (L) at (3,0) {Losange};
%%%%%\node[quadri] (C) at (5,-1.5) {Carré};
%%%%%% les flèches
%%%%%\draw[estun] (P)--(Q);
%%%%%\draw[estun] (R)to[bend left](Q); \draw[estun] (R)--(P);
%%%%%\draw[estun] (L)--(Q.south east); \draw[estun] (L)--(P);
%%%%%\draw[estun] (C)to[bend right](Q.east); \draw[estun] (C)to[bend left](P);
%%%%%\draw[estun] (C)--(L.south east); \draw[estun] (C)to[bend left](R);
%%%%%% la légende
%%%%%\draw[estun] (-4.5,2.5)--(-3,2.5)node[midway,above]{est un};
%%%%%\end{tikzpicture}
