Le quadrilatère $ABCD$ ci-dessous a été dessiné dans un repère orthonormé qui a disparu.
Retrouvez le repère initial à partir des coordonnées des points : $\pointcoord{A}{-4}{2}$, $\pointcoord{B}{2}{-6}$,  $\pointcoord{C}{3}{6}$ et $\pointcoord{D}{1}{2}$.


\begin{tikzpicture}[scale=1,every node/.style={scale=1}]
	\begin{scope}[scale=0.9,rotate=50]
		\clip (1,0) circle (7);
		\placerpoint{A}{-4}{2}{below left};
		\placerpoint{B}{2}{-6}{below right};
		\placerpoint{C}{3}{6}{above left};
		\placerpoint{D}{1}{2}{below right};
		\draw (A) -- (B) -- (C) -- (D) -- cycle;
		% Correction
%		\draw[dashed,red] (-10,-20) -- (D);
%		\draw[dashed,red] (10,20) -- (C);
%		\draw[red] (-10,0) -- (10,0)
%			node[rotate=50,sloped,pos=0.55]{$|$}
%			node[rotate=50,sloped,pos=0.55,below right]{$I$};
%		\draw[red] (0,-10) -- (0,10)
%			node[rotate=50,sloped,pos=0.55]{$|$}
%			node[rotate=50,pos=0.55,above left]{$J$};
%		\draw[red] (0,0)
%			node[rotate=50,below left]{$O$};
	\end{scope}
\end{tikzpicture}